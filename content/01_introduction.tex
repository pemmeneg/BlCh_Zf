%! Author = Philipp Emmenegger
%! Date = 30/06/2021

\section{Einführung}
\textbf{Nutzen ComBau}
\begin{itemize}
    \item Programmiersprachen und Sprachkonzepte besser verstehen
    \item Sprachfeatures beurteilen können
    \item Konzepte in verwandten Bereichen einsetzen
\end{itemize}

\subsection{Begriffe}
\textbf{Compiler}
\begin{itemize}
    \item Transformiert Quellcode in Maschinencode
\end{itemize}
\textbf{Runtime System}
\begin{itemize}
    \item Unterstützt die Programmausführung mit Software und Hardware Mechanismen
\end{itemize}
\textbf{Syntax}
\begin{itemize}
    \item Definiert Struktur des Programms
    \item Bewährte Formalismen für Syntax
\end{itemize}
\textbf{Semantik}
\begin{itemize}
    \item Definiert Bedeutung des Programms
    \item Meist in Prosa beschrieben
\end{itemize}

\subsection{Architekturen}
\begin{center}
    \includegraphics[width=0.6\linewidth]{/architekturen.png} 
\end{center}

\subsection{Aufbau Compiler}
\begin{center}
    \includegraphics[width=0.5\linewidth]{/aufbau_compiler.png} 
\end{center}
\subsubsection{Lexer}
\textbf{Lexikalische Analyse, Scanner}
\begin{itemize}
    \item Zerlegt Programmtext in Terminalsymbole (Tokens)
    \item keine Tiefenstruktur
\end{itemize}
\subsubsection{Parser}
\textbf{Syntaktische Analyse}
\begin{itemize}
    \item Erzeugt Syntaxbaum gemäss Programmstruktur
    \item Kontextfreie Sprache
\end{itemize}
\subsubsection{Semantic Checker}
\textbf{Semantische Analyse}
\begin{itemize}
    \item Löst Symbole auf
    \item Prüft Typen und semantische Regeln
\end{itemize}
\subsubsection{Optimization}
\begin{itemize}
    \item Wandelt Zwischendarstellung in effizientere um
\end{itemize}
\subsubsection{Code Generation}
\begin{itemize}
    \item Erzeugt ausführbarer Maschinencode
\end{itemize}
\subsubsection{Zwischenarstellung}
\textbf{Intermediate Representation}
\begin{itemize}
    \item Beschreibt Programm als Datenstruktur (diverse Varianten)
\end{itemize}

\subsection{Aufbau Laufzeitsystem}
\begin{center}
    \includegraphics[width=0.4\linewidth]{/aufbau_laufzeitsystem.png} 
\end{center}
\subsubsection{Loader}
\begin{itemize}
    \item Lädt Maschinencode in Speicher
    \item Veranlasst Ausführung
\end{itemize}
\subsubsection{Interpreter}
\begin{itemize}
    \item Liest Instruktionen und emuliert diese in Software
\end{itemize}
\subsubsection{JIT (Just-In-Time) Compiler}
\begin{itemize}
    \item Übersetzt Code-Teile in Hardware-Instruktionscode
\end{itemize}
\subsubsection{HW-Ausführung (nativ)}
\begin{itemize}
    \item Lässt Instruktionscode direkt auf HW-Prozessor laufen
\end{itemize}
\subsubsection{Metadaten, Heap + Stacks}
\begin{itemize}
    \item Merken Programminfos, Objekte und Prozeduraufrufe
\end{itemize}
\subsubsection{Garbage Collection}
\begin{itemize}
    \item Räumt nicht erreichbare Objecte ab
\end{itemize}
\newpage

\subsection{Syntax}
\subsubsection{EBNF}
\textbf{Extended Backus-Naur Form}
\begin{center}
    \includegraphics[width=0.5\linewidth]{/ebnf.png} 
    \includegraphics[width=0.5\linewidth]{/ebnf_regeln.png} 
\end{center}
\subsubsection{Arithmetische Ausdrücke}
\begin{center}
    \includegraphics[width=0.5\linewidth]{/ebnf_arithmetisch.png} 
\end{center}